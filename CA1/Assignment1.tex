% Options for packages loaded elsewhere
\PassOptionsToPackage{unicode}{hyperref}
\PassOptionsToPackage{hyphens}{url}
%
\documentclass[
]{article}
\usepackage{amsmath,amssymb}
\usepackage{iftex}
\ifPDFTeX
  \usepackage[T1]{fontenc}
  \usepackage[utf8]{inputenc}
  \usepackage{textcomp} % provide euro and other symbols
\else % if luatex or xetex
  \usepackage{unicode-math} % this also loads fontspec
  \defaultfontfeatures{Scale=MatchLowercase}
  \defaultfontfeatures[\rmfamily]{Ligatures=TeX,Scale=1}
\fi
\usepackage{lmodern}
\ifPDFTeX\else
  % xetex/luatex font selection
\fi
% Use upquote if available, for straight quotes in verbatim environments
\IfFileExists{upquote.sty}{\usepackage{upquote}}{}
\IfFileExists{microtype.sty}{% use microtype if available
  \usepackage[]{microtype}
  \UseMicrotypeSet[protrusion]{basicmath} % disable protrusion for tt fonts
}{}
\makeatletter
\@ifundefined{KOMAClassName}{% if non-KOMA class
  \IfFileExists{parskip.sty}{%
    \usepackage{parskip}
  }{% else
    \setlength{\parindent}{0pt}
    \setlength{\parskip}{6pt plus 2pt minus 1pt}}
}{% if KOMA class
  \KOMAoptions{parskip=half}}
\makeatother
\usepackage{xcolor}
\usepackage[margin=1in]{geometry}
\usepackage{longtable,booktabs,array}
\usepackage{calc} % for calculating minipage widths
% Correct order of tables after \paragraph or \subparagraph
\usepackage{etoolbox}
\makeatletter
\patchcmd\longtable{\par}{\if@noskipsec\mbox{}\fi\par}{}{}
\makeatother
% Allow footnotes in longtable head/foot
\IfFileExists{footnotehyper.sty}{\usepackage{footnotehyper}}{\usepackage{footnote}}
\makesavenoteenv{longtable}
\usepackage{graphicx}
\makeatletter
\newsavebox\pandoc@box
\newcommand*\pandocbounded[1]{% scales image to fit in text height/width
  \sbox\pandoc@box{#1}%
  \Gscale@div\@tempa{\textheight}{\dimexpr\ht\pandoc@box+\dp\pandoc@box\relax}%
  \Gscale@div\@tempb{\linewidth}{\wd\pandoc@box}%
  \ifdim\@tempb\p@<\@tempa\p@\let\@tempa\@tempb\fi% select the smaller of both
  \ifdim\@tempa\p@<\p@\scalebox{\@tempa}{\usebox\pandoc@box}%
  \else\usebox{\pandoc@box}%
  \fi%
}
% Set default figure placement to htbp
\def\fps@figure{htbp}
\makeatother
\setlength{\emergencystretch}{3em} % prevent overfull lines
\providecommand{\tightlist}{%
  \setlength{\itemsep}{0pt}\setlength{\parskip}{0pt}}
\setcounter{secnumdepth}{-\maxdimen} % remove section numbering
\usepackage{bookmark}
\IfFileExists{xurl.sty}{\usepackage{xurl}}{} % add URL line breaks if available
\urlstyle{same}
\hypersetup{
  pdftitle={AssignmentI},
  pdfauthor={Silpa Soni Nallacheruvu (19980824-5287) Hernan Aldana (20000526-4999)},
  hidelinks,
  pdfcreator={LaTeX via pandoc}}

\title{AssignmentI}
\author{Silpa Soni Nallacheruvu (19980824-5287) Hernan Aldana
(20000526-4999)}
\date{2024-11-20}

\begin{document}
\maketitle

\section{Summary}\label{summary}

\section{Exercise 1:1}\label{exercise-11}

\subsection{Question 1:}\label{question-1}

Percentage in Favor and Against Legal Abortion by Gender :

The survey results are summarized in a 2×2 table showing responses from
500 women and 600 men regarding their opinions on legal abortion. We
calculate the percentages in favor and against legal abortion separately
for men and women using the formula:

\({\text{Percentage (In Favor)} = \frac{\text{Count (In Favor) of the Gender}}{\text{Total Count of the Gender}} \times 100}\)

\({\text{Percentage (Against)} = \frac{\text{Count (Against) of the Gender}}{\text{Total Count of the Gender}} \times 100}\)

For Women :

\begin{verbatim}
•   In Favor:
\end{verbatim}

\({\text{Percentage (In Favor) for Women} = \frac{309}{500} \times 100 = 61.8\%}\)

\begin{verbatim}
•   Against:
\end{verbatim}

\({\text{Percentage (Against) for Women} = \frac{191}{500} \times 100 = 38.2\%}\)

For Men:

\begin{verbatim}
•   In Favor:
\end{verbatim}

\({\text{Percentage (In Favor) for Men} = \frac{319}{600} \times 100 = 53.2\%}\)

\begin{verbatim}
•   Against:
\end{verbatim}

\({\text{Percentage (Against) for Men} = \frac{281}{600} \times 100 = 46.8\%}\)

Summary:

\begin{longtable}[]{@{}lcc@{}}
\caption{Percentage In Favor and Against Legal Abortion by
Gender}\tabularnewline
\toprule\noalign{}
Gender & \% In Favor & \% Against \\
\midrule\noalign{}
\endfirsthead
\toprule\noalign{}
Gender & \% In Favor & \% Against \\
\midrule\noalign{}
\endhead
\bottomrule\noalign{}
\endlastfoot
Women & 61.8 & 38.2 \\
Men & 53.2 & 46.8 \\
\end{longtable}

From the analysis, a higher percentage of women (61.8\%) support legal
abortion compared to men (53.2\%). Similarly, a larger percentage of men
(46.8\%) are against legal abortion compared to women (38.2\%).

\subsection{Question 2:}\label{question-2}

\subsubsection{Approach :}\label{approach}

Define Hypotheses :

Null Hypothesis (\({H_0}\)) : There is no difference in opinions between
men and women on legal abortion.

Alternative Hypothesis (\({H_A}\)) : There is a difference in opinions
between men and women on legal abortion.

The formula for Pearson's Chi-Squared Statistic (\({X^2}\)) is:

\({X^2 = \sum_{i=1}^{r} \sum_{j=1}^{c} \frac{(O_{ij} - E_{ij})^2}{E_{ij}}}\)

and

The formula for Likelihood Ratio Statistic (\({G^2}\)) is:

\({G^2 = 2 \sum_{i=1}^{r} \sum_{j=1}^{c} O_{ij} \log\left(\frac{O_{ij}}{E_{ij}}\right)}\)

where \({O_{ij}}\) is the Observed Count, \({E_{ij}}\) is the Expected
Count of (i,j) cell with i-th index of X and j-th index of Y, r is the
total number of rows and c is the total number of columns.

The observed counts are given in the 2×2 table where \({O_{11}}\) = 309,
\({O_{12}}\) = 191, \({O_{21}}\) = 319, \({O_{22}}\) = 281.

The r and c values are also implied for a 2×2 table as r = 2 and c = 2.

Let us calculate the Expected Counts for each cell.

Under \({H_0}\), expected counts are calculated using:

\({E_{ij} = \frac{\text{i-th Row Total} \times \text{j-th Column Total}}{\text{Grand Total}}}\)

Using the row and column totals:

\({E_{11} = \frac{500 \times 628}{1100} = 285.45}\)

\({E_{12} = \frac{500 \times 472}{1100} = 214.55}\)

\({E_{21} = \frac{600 \times 628}{1100} = 342.55}\)

\({E_{22} = \frac{600 \times 472}{1100} = 257.45}\)

Let us calculate the each term in Pearson's Chi-Squared Statistic
(\({X^2}\)) :

For cell (1,1): \({\frac{(309 - 285.45)^2}{285.45} = 1.97}\)

For cell (1,2): \({\frac{(191 - 214.55)^2}{214.55} = 2.84}\)

For cell (2,1): \({\frac{(319 - 342.55)^2}{342.55} = 1.71}\)

For cell (2,2): \({\frac{(281 - 257.45)^2}{257.45} = 2.14}\)

Summing these terms:

\({X^2}\) = 1.97 + 2.84 + 1.71 + 2.14 = 8.66

Degrees of freedom is defined by (r -1)(c -1) = (2-1)(2-1) = 1.

Let us calculate the P-value for \({X^2}\) using a chi-squared table,
which gives us :

\({P(X^2 > 8.66) = 0.0032}\)

Let us calculate the each term in Likelihood Ratio Statistic (\({G^2}\))
:

For cell (1,1):
\({2 \times 309 \times \log\left(\frac{309}{285.45}\right) = 15.88}\)

For cell (1,2):
\({2 \times 191 \times \log\left(\frac{191}{214.55}\right) = 24.20}\)

For cell (2,1):
\({2 \times 319 \times \log\left(\frac{319}{342.55}\right) = 14.75}\)

For cell (2,2):
\({2 \times 281 \times \log\left(\frac{281}{257.45}\right) = 24.61}\)

Summing these terms:

\({G^2}\) = 15.88 + 24.20 + 14.75 + 24.61 = 79.44

Degrees of freedom here (r-1)(c-1) = 1 as well.

Let us calculate the P-value for \({G^2}\) using a chi-squared table as
well, which gives us :

\({P(G^2 > 79.44) \approx 0.003}\)

\subsubsection{Conclusion :}\label{conclusion}

We use a significance level \({\alpha}\) = 0.05 to test our null
hypothesis \({H_0}\). We see that both the \({X^2}\) and \({G^2}\)
statistics yield small p-values, much lesser than 0.05. Hence, we reject
the null hypothesis \({H_0}\), which indicates that there is a
significant difference in opinions on legal abortion between men and
women.

\subsection{Question 3:}\label{question-3}

The odds ratio (OR) is a measure used to quantify the strength of
association between two categorical variables. For the given 2×2 table
regarding opinions on legal abortion among men and women, we calculate
the odds ratio and its 95\% confidence interval. Depending on the chosen
event and conditioning, multiple odds ratios can be formulated. Below,
we calculate the odds ratio of being ``in favor'' of legal abortion for
men compared to women.

\subsubsection{Defining the Odds :}\label{defining-the-odds}

Odds for Women:

\({\text{Odds (Women)} = \frac{\text{In Favor (Women)}}{\text{Against (Women)}} = \frac{309}{191} = 1.6183}\)

Odds for Men:

\({\text{Odds (Men)} = \frac{\text{In Favor (Men)}}{\text{Against (Men)}} = \frac{319}{281} = 1.1352}\)

Odds Ratio (OR) :

The odds ratio compares the odds of being ``in favor'' for men to that
for women:

\({\text{OR} = \frac{\text{Odds (Men)}}{\text{Odds (Women)}} = \frac{1.1352}{1.6183} = 0.7015}\)

Alternative Odds Ratios can be formulated depending on the conditioning
variable:

Odds ratio of being a man vs.~a woman among those ``in favor'':

\({\text{OR} = \frac{\text{Men In Favor / Women In Favor}}{\text{Men Against / Women Against}}}\)

Odds ratio of being against legal abortion for men vs.~women:

\({\text{OR} = \frac{\text{Men Against / Women Against}}{\text{Men In Favor / Women In Favor}}}\)

Each odds ratio provides different insights based on the event being
studied and the conditioning variable.

To compute the 95\% confidence interval for the odds ratio, let us use
the logarithm of odds ratio for a more accurate result and then apply
the delta method to get the confidence interval of odds ratio from the
confidence interval of logarithm of odds ratio.

Logarithm of Odds Ratio:

\({\log(\text{OR}) = \log(0.7015) = -0.3544}\)

Standard Error of \({\log(\text{OR})}\):

The formula for the standard error is:

\({\text{SE} = \sqrt{\frac{1}{O_{11}} + \frac{1}{O_{12}} + \frac{1}{O_{21}} + \frac{1}{O_{22}}}}\)

where \({O_{ij}}\) are the observed frequencies from the table for the
(i,j) cell.

Substituting the values:

\({\text{SE} = \sqrt{\frac{1}{309} + \frac{1}{191} + \frac{1}{319} + \frac{1}{281}} = 0.1191}\)

The 95\% confidence interval for \({\log(\text{OR})}\) is given by:

\({\log(\text{OR}) \pm z_{0.975} \times \text{SE}}\)

Using \({z_{0.975} = 1.96}\):

\({-0.3544 \pm 1.96 \times 0.1191 = (-0.5878, -0.1210)}\)

Exponentiating to Get the Confidence Interval for OR:

To get the confidence interval for the odds ratio, exponentiate the
bounds:

\({\text{Lower Bound} = e^{-0.5878} = 0.5560}\)

\({\text{Upper Bound} = e^{-0.1210} = 0.8862}\)

Thus, the 95\% confidence interval for the odds ratio is:

\({(0.556, 0.886)}\)

\subsubsection{Interpretation}\label{interpretation}

The estimated odds ratio is 0.7015. This means that the odds of men
being ``in favor'' of legal abortion are approximately 70\% of the odds
for women. The 95\% confidence interval (0.556, 0.886) does not include
1, indicating that the difference in odds is statistically significant
at the 5\% significance level. The estimated odds ratio indicates that
the women are significantly more likely to support legal abortion
compared to men and the 95\% confidence interval of the odds ratio
support that indication.

\subsection{Question 4:}\label{question-4}

The risk ratio (also called relative risk, RR) quantifies the likelihood
of an event occurring in one group relative to another group. For the
given 2×2 table regarding opinions on legal abortion among men and
women, we calculate the risk ratio and its 95\% confidence interval.
Below, we calculate the risk ratio of being ``in favor'' of legal
abortion for men compared to women.

\subsubsection{Defining the Risk}\label{defining-the-risk}

The risk refers to the probability of being ``in favor'' of legal
abortion, calculated as:

\({\text{Risk} = \frac{\text{Number of Individuals “In Favor”}}{\text{Total Number of Individuals in the Gender}}}\)

Risk for Women:

\({\text{Risk (Women)} = \frac{\text{Number of Women In Favor}}{\text{Total (Women)}} = \frac{309}{500} = 0.618}\)

Risk for Men:

\({\text{Risk (Men)} = \frac{\text{Number of Men In Favor}}{\text{Total (Men)}} = \frac{319}{600} = 0.532}\)

Risk Ratio (Relative Risk) :

The risk ratio (RR) compares the risk of being ``in favor'' for men to
that for women:

\({\text{RR} = \frac{\text{Risk (Men)}}{\text{Risk (Women)}} = \frac{0.532}{0.618} = 0.861}\)

This value indicates that men are about 86.1\% as likely as women to be
``in favor'' of legal abortion.

To compute the 95\% confidence interval for the risk ratio, let us use
the logarithm of risk ratio for a more accurate result, similar to odds
ratio and then apply the delta method to get the confidence interval of
risk ratio from the confidence interval of logarithm of risk ratio.

Logarithm of Risk Ratio:

\({\log(\text{RR}) = \log(0.861) = -0.1493}\)

Standard Error of \({\log(\text{RR})}\) :

The formula for standard error is:

\({\text{SE} = \sqrt{\frac{1}{\text{In Favor (Men)}} - \frac{1}{\text{Total (Men)}} + \frac{1}{\text{In Favor (Women)}} - \frac{1}{\text{Total (Women)}}}}\)

Substituting the values:

\({\text{SE} = \sqrt{\frac{1}{319} - \frac{1}{600} + \frac{1}{309} - \frac{1}{500}} = 0.0655}\)

The 95\% confidence interval for \({\log(\text{RR})}\) is given by:

\({\log(\text{RR}) \pm z_{0.975} \times \text{SE}}\)

Using \({z_{0.975} = 1.96}\) :

\({-0.1493 \pm 1.96 \times 0.0655 = (-0.2788, -0.0198)}\)

Exponentiating to Get the Confidence Interval for RR:

To get the confidence interval for the risk ratio, exponentiate the
bounds:

\({\text{Lower Bound} = e^{-0.2788} = 0.7565}\)

\({\text{Upper Bound} = e^{-0.0198} = 0.9804}\)

Thus, the 95\% confidence interval for the risk ratio is:

(0.756, 0.980)

\subsubsection{Interpretation}\label{interpretation-1}

The estimated risk ratio is 0.861. This means that men are about 86.1\%
as likely as women to be ``in favor'' of legal abortion. The 95\%
confidence interval (0.756, 0.980) does not include 1, indicating that
the difference in risk is statistically significant at the 5\% level.
The estimated risk ratio indicates that the women are significantly more
likely to support legal abortion compared to men and the 95\% confidence
interval of the risk ratio support that indication.

\subsubsection{Comparison with Odds
Ratio}\label{comparison-with-odds-ratio}

Odds Ratio: Previously calculated as 0.7015, quantifying the odds of
being ``in favor'' for men compared to women.

Risk Ratio: Calculated as 0.861, quantifying the relative probability of
being ``in favor.''

The odds ratio measures the ratio of odds, which can overstate the
association, especially when the event probability is high (e.g., large
proportions of people ``in favor''). The risk ratio provides a more
intuitive interpretation as it measures the relative likelihood.

\subsection{Question 5:}\label{question-5}

\subsubsection{Approach :}\label{approach-1}

Prepare the data and calculate row percentages

\begin{verbatim}
##        opinion
## gender  favor against  Sum
##   women   309     191  500
##   men     319     281  600
##   Sum     628     472 1100
\end{verbatim}

\begin{verbatim}
##        opinion
## gender      favor   against       Sum
##   women 0.6180000 0.3820000 1.0000000
##   men   0.5316667 0.4683333 1.0000000
\end{verbatim}

Calculate X2, G2 and p-values

\begin{verbatim}
## 
##  Pearson's Chi-squared test
## 
## data:  tab1
## X-squared = 8.2979, df = 1, p-value = 0.003969
\end{verbatim}

\begin{verbatim}
## Call:
## loglm(formula = ~gender + opinion, data = tab1)
## 
## Statistics:
##                       X^2 df    P(> X^2)
## Likelihood Ratio 8.322320  1 0.003916088
## Pearson          8.297921  1 0.003969048
\end{verbatim}

Calculate odds ratio and 95\% confidence interval

\begin{verbatim}
## $data
##        opinion
## gender  favor against Total
##   women   309     191   500
##   men     319     281   600
##   Total   628     472  1100
## 
## $measure
##        odds ratio with 95% C.I.
## gender  estimate    lower    upper
##   women 1.000000       NA       NA
##   men   1.425085 1.119482 1.814113
## 
## $p.value
##        two-sided
## gender   midp.exact fisher.exact  chi.square
##   women          NA           NA          NA
##   men   0.003990219  0.004071121 0.003969048
## 
## $correction
## [1] FALSE
## 
## attr(,"method")
## [1] "Unconditional MLE & normal approximation (Wald) CI"
\end{verbatim}

reverse the rows : odds of women to men :

\begin{verbatim}
## $data
##        opinion
## gender  favor against Total
##   men     319     281   600
##   women   309     191   500
##   Total   628     472  1100
## 
## $measure
##        odds ratio with 95% C.I.
## gender   estimate     lower     upper
##   men   1.0000000        NA        NA
##   women 0.7017126 0.5512336 0.8932701
## 
## $p.value
##        two-sided
## gender   midp.exact fisher.exact  chi.square
##   men            NA           NA          NA
##   women 0.003990219  0.004071121 0.003969048
## 
## $correction
## [1] FALSE
## 
## attr(,"method")
## [1] "Unconditional MLE & normal approximation (Wald) CI"
\end{verbatim}

reverse the column : odds of men to women :

\begin{verbatim}
## $data
##        opinion
## gender  against favor Total
##   women     191   309   500
##   men       281   319   600
##   Total     472   628  1100
## 
## $measure
##        odds ratio with 95% C.I.
## gender   estimate     lower     upper
##   women 1.0000000        NA        NA
##   men   0.7017126 0.5512336 0.8932701
## 
## $p.value
##        two-sided
## gender   midp.exact fisher.exact  chi.square
##   women          NA           NA          NA
##   men   0.003990219  0.004071121 0.003969048
## 
## $correction
## [1] FALSE
## 
## attr(,"method")
## [1] "Unconditional MLE & normal approximation (Wald) CI"
\end{verbatim}

reverse both : odds of women to men :

\begin{verbatim}
## $data
##        opinion
## gender  against favor Total
##   men       281   319   600
##   women     191   309   500
##   Total     472   628  1100
## 
## $measure
##        odds ratio with 95% C.I.
## gender  estimate    lower    upper
##   men   1.000000       NA       NA
##   women 1.425085 1.119482 1.814113
## 
## $p.value
##        two-sided
## gender   midp.exact fisher.exact  chi.square
##   men            NA           NA          NA
##   women 0.003990219  0.004071121 0.003969048
## 
## $correction
## [1] FALSE
## 
## attr(,"method")
## [1] "Unconditional MLE & normal approximation (Wald) CI"
\end{verbatim}

\section{Exercise 1:2}\label{exercise-12}

\subsection{Question 1:}\label{question-1-1}

\subsubsection{Approach :}\label{approach-2}

We are analyzing the admissions data from the University of California,
Berkeley, following the 2x2 contingency table:

\begin{table}

\caption{\label{tab:unnamed-chunk-7}Admissions Data by Gender}
\centering
\begin{tabular}[t]{lrr}
\toprule
  & admitted & not admitted\\
\midrule
men & 1198 & 1493\\
women & 557 & 1278\\
\bottomrule
\end{tabular}
\end{table}

Our goal is to perform the following analyses: 1. Calculate the
percentages of admitted and not admitted applicants separately. 2. Test
for independence between gender and admission using: - Pearson's
Chi-Squared Test - Likelihood Ratio Test 3. Calculate the odds ratio and
its 95\% confidence interval. 4. Calculate the risk ratio and its 95\%
confidence interval.

\begin{enumerate}
\def\labelenumi{\arabic{enumi}.}
\tightlist
\item
  Calculating Percentages
\end{enumerate}

For men: - Total men: \$ n\_\{\text{Men}\} = 2691 \$ - Admitted men: \$
a\_\{\text{Men}\} = 1198 \$ - Not admitted men: \$ n\_\{\text{Men}\} -
a\_\{\text{Men}\} = 1493 \$

Percentages

\text{Percentage Admitted (Men)} = \left( \frac{1198}{2691} \right)
\times 100\% \approx 44.53\% \text{Percentage Not Admitted (Men)} =
100\% - 44.53\% = 55.47\%

For women: - Total women: \$ n\_\{\text{Women}\} = 1835 \$ - Admitted
women: \$ a\_\{\text{Women}\} = 557 \$ - Not admitted women: \$
n\_\{\text{Women}\} - a\_\{\text{Women}\} = 1278 \$

Percentages

\text{Percentage Admitted (Women)} = \left( \frac{557}{1835} \right)
\times 100\% \approx 30.35\% \text{Percentage Not Admitted (Women)} =
100\% - 30.35\% = 69.65\%

\begin{table}

\caption{\label{tab:unnamed-chunk-8}Admissions Data by Gender}
\centering
\begin{tabular}[t]{lll}
\toprule
  & admitted & not admitted\\
\midrule
men & 44.53\% & 55.47\%\\
women & 30.35\% & 69.65\%\\
\bottomrule
\end{tabular}
\end{table}

\begin{verbatim}
## 
##  Pearson's Chi-squared test
## 
## data:  tab2
## X-squared = 92.205, df = 1, p-value < 2.2e-16
\end{verbatim}

\begin{verbatim}
## Call:
## loglm(formula = ~gender + admission, data = tab2)
## 
## Statistics:
##                       X^2 df P(> X^2)
## Likelihood Ratio 93.44941  1        0
## Pearson          92.20528  1        0
\end{verbatim}

\begin{verbatim}
## $data
##        admission
## gender  admitted not admitted Total
##   men       1198         1493  2691
##   women      557         1278  1835
##   Total     1755         2771  4526
## 
## $measure
##        odds ratio with 95% C.I.
## gender  estimate    lower    upper
##   men    1.00000       NA       NA
##   women  1.84108 1.624377 2.086693
## 
## $p.value
##        two-sided
## gender  midp.exact fisher.exact chi.square
##   men           NA           NA         NA
##   women          0 4.835903e-22 7.8136e-22
## 
## $correction
## [1] FALSE
## 
## attr(,"method")
## [1] "Unconditional MLE & normal approximation (Wald) CI"
\end{verbatim}

\begin{verbatim}
## $data
##        admission
## gender  admitted not admitted Total
##   men       1198         1493  2691
##   women      557         1278  1835
##   Total     1755         2771  4526
## 
## $measure
##        risk ratio with 95% C.I.
## gender  estimate    lower   upper
##   men   1.000000       NA      NA
##   women 1.255303 1.199631 1.31356
## 
## $p.value
##        two-sided
## gender  midp.exact fisher.exact chi.square
##   men           NA           NA         NA
##   women          0 4.835903e-22 7.8136e-22
## 
## $correction
## [1] FALSE
## 
## attr(,"method")
## [1] "Unconditional MLE & normal approximation (Wald) CI"
\end{verbatim}

\end{document}
